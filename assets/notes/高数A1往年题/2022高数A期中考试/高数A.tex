\documentclass{article}
\usepackage{ctex}
\usepackage{amsmath}
\begin{document}
	\begin{center}
		高数A
	\end{center}
一、
\\(1)有理数的有理数次幂是否一定是有理数?
\\(2)无理数的无理数次幂是否一定是无理数?
\\二、
\\(1)$\lim\limits_{x \to +\infty} \frac{\sqrt{x+2\sqrt{x+2\sqrt{x}}}}{\sqrt{x+4}}$
\\(2)$\lim\limits_{x \to 1}(\frac{1}{x-1}-\frac{2}{x^2-1}+\frac{3}{x^3-1}-\frac{4}{x^4-1})$
\\(3)$\lim\limits_{x \to +\infty} (2021\sqrt{x+2021}+2023\sqrt{x+2023}-2\cdot2022\sqrt{x+2022})$
\\三、
\\(1)$f'(x)=1+e^{-x}$,求$f(x)$
\\(2)$g'(x)+g(x)=1+e^{-x}$,求$g(x)$
\\四、
\\是否存在实数序列$\left \{ a_{n} \right \} ^{+\infty}_{n=1}$使得$\lim\limits_{n\to +\infty}a_{n}=1,\lim\limits_{n\to +\infty}a_{n}^{n}=1.001$?
\\五、
\\(1)$y=(x^2+2x+2)e^{-x}$,求$y^{(n)}$
\\(2)$y=\int_{\cot{x}}^{\tan{x}}\sqrt{1+t^2}dt$,求$y'$
\\六、
\\已知$f(x)$,$x\in[a,b]$
\\(1)若$f(x)$在$x_{0}$处可导,证明$f(x)$在其一个小邻域内连续
\\(2)若$f(x)$在$x_{0}$处二阶可导,证明$f(x)$在其一个小邻域内连续
\\七、
\\$f(x)\in C[a,b],f(x) \in R[a,b],F(x)=\int_{a}^{x}f(t)dt$,证明:
\\(1)$F(x) \in C[a,b]$
\\(2)$F(x)$在$[a,b]$上可导,且$\forall x \in (a,b),F'(x)=f(x)$
\\八、
\\$a_1=\sqrt{2},a_{n+1}=\sqrt{2}^{a_n},\text{判断 }\lim\limits_{n\to +\infty}a_n \text{是否存在,若存在求出其值,若不存在说明理由}$
\\九、
\\求一个$\left\{{\xi_{n}}\right\}$使得$(i),(ii)$成立
\\$(i) \lim\limits_{n\to +\infty}(\xi_n-e^n)=0;(ii)\lim\limits_{n\to +\infty}(f(\xi_n)-f(e^n))\neq0;\text{其中}f(x)=x\ln x,x\in (0,+\infty)$
\end{document}